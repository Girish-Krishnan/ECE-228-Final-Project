\documentclass{article}


% if you need to pass options to natbib, use, e.g.:
%     \PassOptionsToPackage{numbers, compress}{natbib}
% before loading neurips_2022


% ready for submission
\usepackage[preprint]{neurips_2022}


% to compile a preprint version, e.g., for submission to arXiv, add add the
% [preprint] option:
%     \usepackage[preprint]{neurips_2022}


% to compile a camera-ready version, add the [final] option, e.g.:
%     \usepackage[final]{neurips_2022}


% to avoid loading the natbib package, add option nonatbib:
%    \usepackage[nonatbib]{neurips_2022}


\usepackage[utf8]{inputenc} % allow utf-8 input
\usepackage[T1]{fontenc}    % use 8-bit T1 fonts
\usepackage{hyperref}       % hyperlinks
\usepackage{url}            % simple URL typesetting
\usepackage{booktabs}       % professional-quality tables
\usepackage{amsfonts}       % blackboard math symbols
\usepackage{nicefrac}       % compact symbols for 1/2, etc.
\usepackage{microtype}      % microtypography
\usepackage{xcolor}         % colors
\newcommand{\new}[1]{{\color{red} #1}}

\title{Project Template For ECE228}


% The \author macro works with any number of authors. There are two commands
% used to separate the names and addresses of multiple authors: \And and \AND.
%
% Using \And between authors leaves it to LaTeX to determine where to break the
% lines. Using \AND forces a line break at that point. So, if LaTeX puts 3 of 4
% authors names on the first line, and the last on the second line, try using
% \AND instead of \And before the third author name.


\author{%
  First Name, Last Name
  % examples of more authors
  % \And
  % Coauthor \\
  % Affiliation \\
  % Address \\
  % \texttt{email} \\
  % \AND
  % Coauthor \\
  % Affiliation \\
  % Address \\
  % \texttt{email} \\
  % \And
  % Coauthor \\
  % Affiliation \\
  % Address \\
  % \texttt{email} \\
  % \And
  % Coauthor \\
  % Affiliation \\
  % Address \\
  % \texttt{email} \\
}


\begin{document}


\maketitle


% \begin{abstract}
% \end{abstract}
\noindent Please submit your proposal as a single PDF file using this template. Use the template as follows:
\begin{itemize}
    \item Make a copy of this template (do NOT edit) as a new Overleaf project.
    \item Change title to \textbf{ECE228 Project Proposal: Your Title}.
    \item Change author name and email to your  team members' names and emails.
    \item Follow the format instructions in Section \ref{sec:1} to write your proposal.
    \item Delete the instructions in your final submitted report.
\end{itemize}


\section{Project Proposal: Track 1}\label{sec:1}

Your proposal should include:
\begin{itemize}
    \item Problem background \& Motivation
    \item Related Works [please include at least one reference paper published within the past 5 years that you treat as baseline for Track 1 or a link to a dataset for Track 2]
    \item High-level Methodology
\end{itemize}




\end{document}
