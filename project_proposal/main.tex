\documentclass{article}


% if you need to pass options to natbib, use, e.g.:
%     \PassOptionsToPackage{numbers, compress}{natbib}
% before loading neurips_2022


% ready for submission
\usepackage[preprint]{neurips_2022}


% to compile a preprint version, e.g., for submission to arXiv, add add the
% [preprint] option:
%     \usepackage[preprint]{neurips_2022}


% to compile a camera-ready version, add the [final] option, e.g.:
%     \usepackage[final]{neurips_2022}


% to avoid loading the natbib package, add option nonatbib:
%    \usepackage[nonatbib]{neurips_2022}


\usepackage[utf8]{inputenc} % allow utf-8 input
\usepackage[T1]{fontenc}    % use 8-bit T1 fonts
\usepackage{hyperref}       % hyperlinks
\usepackage{url}            % simple URL typesetting
\usepackage{booktabs}       % professional-quality tables
\usepackage{amsfonts}       % blackboard math symbols
\usepackage{nicefrac}       % compact symbols for 1/2, etc.
\usepackage{microtype}      % microtypography
\usepackage{xcolor}         % colors
\newcommand{\new}[1]{{\color{red} #1}}

\title{ECE228 Project Proposal: Benchmarking Physics-Informed Models for Full Waveform Inversion}


% The \author macro works with any number of authors. There are two commands
% used to separate the names and addresses of multiple authors: \And and \AND.
%
% Using \And between authors leaves it to LaTeX to determine where to break the
% lines. Using \AND forces a line break at that point. So, if LaTeX puts 3 of 4
% authors names on the first line, and the last on the second line, try using
% \AND instead of \And before the third author name.


\author{%
  Girish Krishnan \\
  gikrishnan@ucsd.edu
  % examples of more authors
  \And
  Ryan Irwandy \\
  rirwandy@ucsd.edu
  \And
  Yash Puneet \\
  ypuneet@ucsd.edu
  \And
  Harini Gurusankar \\
  hgurusan@ucsd.edu
  % Coauthor \\
  % Affiliation \\
  % Address \\
  % \texttt{email} \\
  % \AND
  % Coauthor \\
  % Affiliation \\
  % Address \\
  % \texttt{email} \\
  % \And
  % Coauthor \\
  % Affiliation \\
  % Address \\
  % \texttt{email} \\
  % \And
  % Coauthor \\
  % Affiliation \\
  % Address \\
  % \texttt{email} \\
}


\begin{document}


\maketitle


% \begin{abstract}
% \end{abstract}
% \noindent Please submit your proposal as a single PDF file using this template. Use the template as follows:
% \begin{itemize}
%     \item Make a copy of this template (do NOT edit) as a new Overleaf project.
%     \item Change title to \textbf{ECE228 Project Proposal: Your Title}.
%     \item Change author name and email to your  team members' names and emails.
%     \item Follow the format instructions in Section \ref{sec:1} to write your proposal.
%     \item Delete the instructions in your final submitted report.
% \end{itemize}


% \section{Project Proposal: Track 1}\label{sec:1}

% Your proposal should include:
% \begin{itemize}
%     \item Problem background \& Motivation
%     \item Related Works [please include at least one reference paper published within the past 5 years that you treat as baseline for Track 1 or a link to a dataset for Track 2]
%     \item High-level Methodology
% \end{itemize}

\vspace{-1em}
\section{Problem Background \& Motivation}

Full waveform inversion (FWI) is a computational method used in geophysics to infer subsurface properties by minimizing the difference between observed and simulated seismic waveforms. Traditional FWI relies on solving partial differential equations (PDEs) iteratively, which is computationally intensive and sensitive to initial conditions. Recent machine learning methods aim to reduce computational costs and improve robustness but face challenges in generalizing to varying source parameters and handling noisy or incomplete data. Addressing these issues is essential for reliable subsurface imaging in applications such as energy exploration and earthquake hazard assessment.

\section{Related Works}

The Fourier-DeepONet model \citep{zhu2023fourier} enhances the DeepONet architecture with Fourier features, enabling generalization across seismic source conditions and improving accuracy over earlier data-driven FWI methods. InversionNet \citep{wu2018inversionnet} uses a convolutional neural network (CNN) with a conditional random field (CRF) to map seismic data to subsurface velocity models, reducing computational cost while embedding spatial and temporal constraints. Neural ODEs \citep{chen2018neural} model hidden states through differential equations, offering continuous-depth representation with constant memory use. The PCA-SOM approach \citep{zhang2019pca} combines principal component analysis (PCA) for dimensionality reduction with self-organizing maps (SOMs) for clustering, effectively handling high-dimensional, redundant geophysical data. While these models work well in controlled settings, their performance on complex, real-world datasets like the Kaggle Geophysical Waveform Inversion competition \citep{kaggle2025} remains uncertain. Challenges in generalization, noise robustness, and computational efficiency motivate us to explore combining the strengths of multiple approaches to surpass individual baselines.

\section{High-Level Methodology}

We will reimplement and benchmark four baseline approaches on the Kaggle competition dataset: Fourier-DeepONet, InversionNet, Neural ODE, and PCA-SOM. We will adapt the Fourier-DeepONet architecture to predict velocity maps, implement InversionNet using CNNs and CRFs, develop a Neural ODE model using a differential equation solver, and apply the PCA-SOM pipeline for dimensionality reduction and clustering. After evaluating baseline performance using metrics such as mean absolute error (MAE), mean squared error (MSE), and structural similarity index (SSIM), we will explore improvements by combining elements across models, such as integrating Fourier features or neural operators into InversionNet or adding Neural ODE layers into DeepONet. We aim to identify whether such hybrid designs can improve accuracy, generalization, and efficiency. We will submit the best-performing models to the Kaggle competition to benchmark effectiveness and aim for top leaderboard performance, while systematically analyzing how combined techniques advance physics-informed FWI modeling.

\bibliographystyle{plainnat}
\bibliography{references}



\end{document}
