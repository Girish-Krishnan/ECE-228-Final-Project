\documentclass{article}


% if you need to pass options to natbib, use, e.g.:
%     \PassOptionsToPackage{numbers, compress}{natbib}
% before loading neurips_2022


% ready for submission
\usepackage[preprint]{neurips_2022}


% to compile a preprint version, e.g., for submission to arXiv, add add the
% [preprint] option:
%     \usepackage[preprint]{neurips_2022}


% to compile a camera-ready version, add the [final] option, e.g.:
%     \usepackage[final]{neurips_2022}


% to avoid loading the natbib package, add option nonatbib:
%    \usepackage[nonatbib]{neurips_2022}


\usepackage[utf8]{inputenc} % allow utf-8 input
\usepackage[T1]{fontenc}    % use 8-bit T1 fonts
\usepackage{hyperref}       % hyperlinks
\usepackage{url}            % simple URL typesetting
\usepackage{booktabs}       % professional-quality tables
\usepackage{amsfonts}       % blackboard math symbols
\usepackage{nicefrac}       % compact symbols for 1/2, etc.
\usepackage{microtype}      % microtypography
\usepackage{xcolor}         % colors
\newcommand{\new}[1]{{\color{red} #1}}

\title{ECE228 Project Proposal:  Physics-Informed Neural Networks for Soft Robot Kinematics}


% The \author macro works with any number of authors. There are two commands
% used to separate the names and addresses of multiple authors: \And and \AND.
%
% Using \And between authors leaves it to LaTeX to determine where to break the
% lines. Using \AND forces a line break at that point. So, if LaTeX puts 3 of 4
% authors names on the first line, and the last on the second line, try using
% \AND instead of \And before the third author name.


\author{%
  Girish Krishnan \\
  gikrishnan@ucsd.edu
  % examples of more authors
  \And
  Ryan Irwandy \\
  rirwandy@ucsd.edu
  \And
  Yash Puneet \\
  ypuneet@ucsd.edu
  \And
  Harini Gurusankar \\
  hgurusan@ucsd.edu
  % Coauthor \\
  % Affiliation \\
  % Address \\
  % \texttt{email} \\
  % \AND
  % Coauthor \\
  % Affiliation \\
  % Address \\
  % \texttt{email} \\
  % \And
  % Coauthor \\
  % Affiliation \\
  % Address \\
  % \texttt{email} \\
  % \And
  % Coauthor \\
  % Affiliation \\
  % Address \\
  % \texttt{email} \\
}


\begin{document}


\maketitle


% \begin{abstract}
% \end{abstract}
% \noindent Please submit your proposal as a single PDF file using this template. Use the template as follows:
% \begin{itemize}
%     \item Make a copy of this template (do NOT edit) as a new Overleaf project.
%     \item Change title to \textbf{ECE228 Project Proposal: Your Title}.
%     \item Change author name and email to your  team members' names and emails.
%     \item Follow the format instructions in Section \ref{sec:1} to write your proposal.
%     \item Delete the instructions in your final submitted report.
% \end{itemize}


% \section{Project Proposal: Track 1}\label{sec:1}

% Your proposal should include:
% \begin{itemize}
%     \item Problem background \& Motivation
%     \item Related Works [please include at least one reference paper published within the past 5 years that you treat as baseline for Track 1 or a link to a dataset for Track 2]
%     \item High-level Methodology
% \end{itemize}

\vspace{-1em}

\section{Problem Background \& Motivation}

Soft robots offer advantages in adaptability and safety due to their compliant structures. However, accurately modeling their kinematics is challenging because of their nonlinear deformations and complex material properties. Traditional modeling approaches often struggle to capture these dynamics, especially when dealing with diverse actuation mechanisms like pneumatic and tendon-driven systems. Developing models that can generalize across different actuation types is crucial for advancing control strategies and real-world applications of soft robotics.

\section{Related Works}

The dataset titled \textit{Positional Data of Soft Robots with Diverse Actuation Types} provides positional data for three types of soft robots: simulated pneumatic, simulated tendon-driven, and real-world tendon-driven robots \citep{ieee_dataset}. This dataset is designed to aid in developing machine learning models for learning the kinematics of soft robots with various actuation types.

Recent studies have explored the use of physics-informed neural networks (PINNs) to model the complex deformations of soft robotic systems. For instance, the PINN-Ray model integrates the principles of elastic mechanics into the neural network's loss function to accurately predict the behavior of Fin Ray soft robotic grippers \citep{pinn_ray}. Additionally, research has demonstrated the application of PINNs in modeling and controlling complex robotic systems, highlighting their potential in handling non-conservative effects and improving control performance \citep{pinn_control}.

\section{High-Level Methodology}

Our project aims to develop physics-informed neural network models that can accurately predict the kinematics of soft robots with diverse actuation mechanisms. The approach involves:

\begin{itemize}
    \item Utilizing the provided dataset to train PINNs that incorporate physical laws governing soft robot behavior.
    \item Designing neural network architectures that embed the minimum potential energy principle from elastic mechanics into their loss functions.
    \item Evaluating the models' performance in predicting positional data for both interpolation and extrapolation scenarios.
    \item Comparing the effectiveness of PINNs against traditional data-driven models in terms of accuracy and generalization capabilities.
\end{itemize}

Through this methodology, we aim to demonstrate the advantages of integrating physical principles into neural network models for soft robot kinematics.


\bibliographystyle{plainnat}
\bibliography{references}



\end{document}
